\documentclass[12pt]{article}
\oddsidemargin 0em
\evensidemargin 0em
\topmargin 0em
\textwidth 40em
\parindent 0pt

\begin{document}

\subsubsection*{Drawing Trees and Animating Tree Changes}

Sandro Badame, Peter Boothe$^*$, Manhattan College
\medskip

Drawing trees and animating tree changes (subtree addition, replacement, and
removal) can allow new visualizations of computational and graph theoretic
processes.  We will be demonstrating a library that we have designed and built
for exactly this purpose as well as two visualizations designed to show off its
capabilities.  Our first demonstration shows the ability to watch the execution
of a piece of Lisp code, while our second example shows a tree being parsed by
a tree automaton.  In the Lisp example, the code being executed may be entered
by the user, and in the tree example the tree is user-entered, although the
automaton is not.  Drawing a graph as a tree can even prove useful when the
graph being drawn is neither directed nor acyclic.  We present one last
example visualization using a graph which is not a DAG, which allows us to
animate the creation and usage of a doubly-linked list.  We hope that the
library will be of use to educators who want to provide a visualization of
trees and graphs and of computational processes on these structures.  
\medskip

Keywords:  computing, teaching, graph visualization, trees, animation
\end{document}
